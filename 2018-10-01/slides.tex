\documentclass[svgnames]{beamer}
\usepackage{mathpartir, stmaryrd}
\usepackage{wasysym}
\usepackage{listings}
\usepackage{multicol}
\usepackage{amssymb, amsthm}
\usepackage{microtype}


\usefonttheme[onlymath]{serif}
\definecolor{JadeGreen}{RGB}{0,168,107}
\definecolor{MunsellPurple}{RGB}{159,0,197}

\definecolor{CobaltBlue}{RGB}{0,71,171}
\definecolor{FireBrick}{RGB}{228,34,23}
\definecolor{Alabaster}{RGB}{250,250,250}

\setbeamertemplate{navigation symbols}{}
\setbeamertemplate{headline}{}
\setbeamercolor{example text}{fg=CobaltBlue}
\setbeamertemplate{itemize items}{$\bullet$}

\newcommand\cxlen[1]{\vert\fmttm{#1}\vert}
\newcommand\fmttm[1]{{\color{Blue}#1}}
\newcommand\fmtval[1]{{\color{Red}#1}}
\newcommand\fmtne[1]{{\color{DarkOrange}#1}}
\newcommand\fmtnf[1]{{\color{Purple}#1}}
\newcommand\fmtclo[1]{{\color{Green}#1}}

\newcommand{\bnfeq}{::=}
\newcommand{\defeq}{\triangleq}
\newcommand{\Uni}{\mathcal{U}}
\newcommand{\Unit}{\mathsf{Unit}}
\newcommand{\unit}{\mathsf{tt}}
\newcommand{\var}[1]{\fmttm{\mathbf{x}_{\color{Black} #1}}}
\newcommand{\nevar}[1]{\fmtne{\mathbf{x}_{\color{Black} #1}}}
\newcommand{\emp}{()}
\newcommand{\pow}[1]{\mathcal{P}(#1)}
\newcommand{\Term}{\mathbf{Term}}
\newcommand{\Type}{\mathbf{Type}}
\newcommand{\Val}{\mathbf{Val}}
\newcommand{\Cxt}{\mathbf{Cxt}}
\newcommand{\pto}{\rightharpoonup}
\newcommand{\mto}{\xrightarrow{\mathsf{mon}}}
\newcommand{\isctx}[1]{\fmttm{#1}\vdash}
\newcommand{\isterm}[3]{\fmttm{#1}\vdash\fmttm{#2}:\fmttm{#3}}
\newcommand{\isneutral}[3]{\fmttm{#1}\vdash^{\mathsf{neu}}\fmtne{#2}:\fmttm{#3}}
\newcommand{\isnormal}[3]{\fmttm{#1}\vdash^{\mathsf{nf}}\fmtnf{#2}:\fmttm{#3}}
\newcommand{\eqterm}[4]{\fmttm{#1} \vdash \fmttm{#2} \equiv \fmttm{#3} : \fmttm{#4}}
\newcommand{\istype}[2]{\fmttm{#1}\vdash\fmttm{#2}}
\newcommand{\eqtype}[3]{\fmttm{#1}\vdash\fmttm{#2} \equiv \fmttm{#3}}
\newcommand{\subst}[3]{{\color{Black} \fmttm{#1}\{\fmttm{#2}/#3\}}}
\newcommand{\weaken}[2]{{\color{Black} \fmttm{#1}\{\uparrow^{#2}\}}}
\newcommand{\app}[2]{{\color{Black} \fmttm{#1}(\fmttm{#2})}}
\newcommand{\neapp}[2]{{\color{Black} \fmtne{#1}(\fmtnf{#2})}}
\newcommand{\nf}[2]{{\color{Black} \mathsf{nf}(\fmttm{#1}, \fmttm{#2})}}

\title{Normalization by Evaluation for \\Martin-L{\"of} Type Theory}
\author{Daniel Gratzer}
\date{\today}

\begin{document}
\begin{frame}
  \titlepage
\end{frame}

\begin{frame}
  \frametitle{Goal}

  Produce a function $\nf{t}{A} : \Term \times \Term \pto \Term$ so that the following 3 conditions
  hold:
  \begin{enumerate}
  \item $\eqterm{\Gamma}{t_1}{t_2}{A} \iff \nf{t_1}{A} = \nf{t_2}{A}$
  \item If $\isterm{\Gamma}{t}{A}$ then $\eqterm{\Gamma}{t}{\nf{t}{A}}{A}$
  \item If $\isterm{\Gamma}{t}{A}$ then $\nf{t}{A}$ is a \emph{normal form}\\ -- more on this shortly.
  \end{enumerate}
  \pause
  \centering
  \emph{N.B. is partial as $\Term$ contains ill-typed terms.}
\end{frame}

\begin{frame}
  \frametitle{Why Bother}
  \centering
  Why bother to do this when it's so much easier to not do things?
  \bigskip
  \begin{enumerate}
  \item Termination, canonicity, consistency are corollaries
  \item Decidability of type-checking\\
    \pause
    This because of the \emph{conversion rule}:
    \[
      \inferrule{
        \eqtype{\Gamma}{A}{B}\\
        \isterm{\Gamma}{t}{A}
      }{
        \isterm{\Gamma}{t}{B}
      }
    \]
  \end{enumerate}
\end{frame}

\begin{frame}
  \frametitle{Why Normalization by Evaluation (NbE)?}
  Techniques for proving normalization abound, why NbE?
  \begin{enumerate}
  \item Scales to support many languages
    \begin{itemize}
    \item full dependent types
    \item proof irrelevant types
    \item impredicative quantification
    \item sized types
    \item (conjectured) guarded dependent type theory
    \item (conjectured) cubical type theory.
    \end{itemize}
  \item Amenable to formalization in a (stronger) type theory
  \item Practical for implementation*
  \item Principled semantic interpretation
  \end{enumerate}
\end{frame}

\begin{frame}
  \frametitle{A Language}
  \begin{mathpar}
    \inferrule{ }{\isctx{\emp}}\and
    \inferrule{
      \isctx{\Gamma}\\
      \istype{\Gamma}{A}
    }{\isctx{\Gamma.A}}\\
    \inferrule{
      \istype{\Gamma}{A}\\
      \istype{\Gamma.A}{B}
    }{\istype{\Gamma}{A \to B}}\and
    \inferrule{
      \isctx{\Gamma}
    }{\istype{\Gamma}{\Unit} \\ \istype{\Gamma}{\Uni}}\and
    \inferrule{
      \isterm{\Gamma}{A}{\Uni}
    }{\istype{\Gamma}{A}}\\
    \inferrule{
      \isctx{\Gamma}
    }{
      \isterm{\Gamma}{\Unit}{\Uni}\\
      \isterm{\Gamma}{\unit}{\Unit}
    }\and
    \inferrule{
      \istype{\Gamma}{A}\\
      \isterm{\Gamma.A}{t}{B}
    }{\isterm{\Gamma}{\lambda t}{A \to B}}\and
    \inferrule{
      \isterm{\Gamma}{t}{A \to B}\\
      \isterm{\Gamma}{u}{A}
    }{\isterm{\Gamma}{\app{t}{u}}{\subst{B}{u}{0}}}\and
    \inferrule{
      \isctx{\Gamma}
    }{\isterm{\Gamma}{\var{n}}{\weaken{\color{Black} \fmttm{\Gamma}(n)}{n}}}
  \end{mathpar}
\end{frame}

\begin{frame}
  \frametitle{Conventions}
  We use De Bruijn indices for representing binding. This simplifies things with NbE later.
  \begin{align*}
    \weaken{t}{i} &\quad \text{``Increment every variable in $t$ by $i$''}\\
    \subst{t}{u}{i} &\quad \text{``Replace $\fmttm{\var{i}}$ with $\fmttm{u}$ in $\fmttm{t}$, adjust variables''}
  \end{align*}
\end{frame}

\begin{frame}
  \frametitle{The Wrinkle}
  The rule which causes so much pain.
  \[
    \inferrule{
      \isterm{\Gamma}{t}{A}\\
      \eqtype{\Gamma}{A}{B}
    }{
      \isterm{\Gamma}{t}{B}
    }
  \]
  Dependence means term equality matters for type equality.
  \[
    \inferrule{
      \eqterm{\Gamma}{A}{B}{\Uni}
    }{\eqtype{\Gamma}{A}{B}}
  \]
\end{frame}
\begin{frame}
  \frametitle{The Wrinkle -- The Main Equality Rules}
  \begin{mathpar}
    \inferrule{
      \isterm{\Gamma}{u}{A}\\
      \isterm{\Gamma.A}{t}{B}
    }{\eqterm{\Gamma}{\app{\color{Black}(\fmttm{\lambda t})}{u}}{\subst{t}{u}{0}}{\subst{B}{u}{0}}}\\
    \inferrule{
      \istype{\Gamma}{A}\\
      \isterm{\Gamma.A}{t}{B}
    }{\eqterm{\Gamma}{\lambda \color{Black}(\app{\weaken{t}{1}}{\var{0}})}{t}{A \to B}}\\
    \inferrule{
      \isterm{\Gamma}{t}{\Unit}
    }{\eqterm{\Gamma}{t}{\unit}{\Unit}}
  \end{mathpar}
\end{frame}

\begin{frame}
  \frametitle{Neutral and Normal Forms}
  Let us isolate special terms which are \emph{canonical} for their equivalence classes. Though this
  can only be observed later.

  \begin{enumerate}
  \item Neutral terms: variables or terms stuck on variables.
  \item Normal forms: terms in $\beta$-normal and $\eta$-long forms.
  \end{enumerate}
  \begin{mathpar}
    \inferrule{
      \isterm{\Gamma}{\var{n}}{A}
    }{\isneutral{\Gamma}{\nevar{n}}{A}}\and
    \inferrule{
      \isneutral{\Gamma}{e}{A \to B}\\
      \isnormal{\Gamma}{v}{A}
    }{\isneutral{\Gamma}{\neapp{e}{v}}{\subst{B}{v}{0}}}\\
    \inferrule{
      \isctx{\Gamma}
    }{
      \isnormal{\Gamma}{\unit}{\Unit}
    }\and
    \inferrule{
      \istype{\Gamma}{A}\\
      \isnormal{\Gamma.A}{t}{B}
    }{\isnormal{\Gamma}{\lambda t}{A \to B}}\and
    \inferrule{
      \isnormal{\Gamma}{A}{\Uni}\\
      \isnormal{\Gamma.A}{B}{\Uni}
    }{\isnormal{\Gamma}{A \to B}{\Uni}}\and
    \inferrule{
      \isneutral{\Gamma}{e}{\Uni}
    }{\isnormal{\Gamma}{e}{\Uni}}\and
  \end{mathpar}
\end{frame}

\begin{frame}
  \frametitle{Normalization by Evaluation}
  \centering
  Now we have a goal, construct $\isnormal{\Gamma}{\nf{t}{A}}{A}$ given $\isterm{\Gamma}{t}{A}$.
\end{frame}
\begin{frame}
  \frametitle{Normalization by Evaluation -- Historical Context}
  \begin{centering}
    Original idea: normalize programs using the ambient semantic universe.
  \end{centering}
  \pause
  \bigskip

  For instance, to evaluate simple types, convert each function to the host languages function and
  then convert back.
\end{frame}

\begin{frame}
  \frametitle{Intrinsic Normalization by Evaluation -- An Example}

\end{frame}
  % \begin{centering}
  %   We will design a domain of \emph{semantic values}. In older presentations of NbE
  % \end{centering}
  % \bigskip

  % \pause
  % Overview of our approach:
  % \begin{enumerate}
  % \item Take a term, $\fmttm{t}$, and \emph{evaluate} it into a semantic value, $\fmtval{v}$
  % \item Quote this semantic value, $\fmtval{v}$, back into a syntactic term, $\fmttm{t'}$
  % \end{enumerate}

\end{document}
