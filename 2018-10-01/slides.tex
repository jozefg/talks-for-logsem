\documentclass[svgnames]{beamer}
\usepackage{mathpartir, stmaryrd}
\usepackage{ulem}
\usepackage{wasysym}
\usepackage{listings}
\usepackage{multicol}
\usepackage{amssymb, amsthm}
\usepackage{microtype}


\usefonttheme[onlymath]{serif}
\definecolor{JadeGreen}{RGB}{0,168,107}
\definecolor{MunsellPurple}{RGB}{159,0,197}

\definecolor{CobaltBlue}{RGB}{0,71,171}
\definecolor{FireBrick}{RGB}{228,34,23}
\definecolor{Alabaster}{RGB}{250,250,250}

\setbeamertemplate{navigation symbols}{}
\setbeamertemplate{headline}{}
\setbeamercolor{example text}{fg=CobaltBlue}
\setbeamertemplate{itemize items}{$\bullet$}

\newcommand\cxtlen[1]{\vert\fmttm{#1}\vert}
\newcommand\fmttm[1]{{\color{Blue}#1}}
\newcommand\fmtval[1]{{\color{Red}#1}}
\newcommand\fmtne[1]{{\color{DarkOrange}#1}}
\newcommand\fmtnf[1]{{\color{Purple}#1}}
\newcommand\fmtclo[1]{{\color{Green}#1}}

\newcommand{\sem}[1]{\llbracket #1 \rrbracket}
\newcommand{\bnfeq}{::=}
\newcommand{\defeq}{\triangleq}
\newcommand{\Uni}{\fmttm{\mathcal{U}}}
\newcommand{\Unit}{\fmttm{\mathsf{Unit}}}
\newcommand{\unit}{\fmttm{\mathsf{tt}}}
\newcommand{\vUnit}{\fmtval{\mathsf{Unit}}}
\newcommand{\vunit}{\fmtval{\mathsf{tt}}}
\newcommand{\vuni}{\fmtval{\mathsf{Uni}}}
\newcommand{\vpi}[2]{\fmtval{\Pi\,\fmtval{#1}{\color{Black} .}\ \fmtclo{#2}}}
\newcommand{\mkclo}[2]{{\color{Black} \fmttm{#1}\{\fmtval{#2}\}}}
\newcommand{\vvar}[1]{{\color{Black} \fmtne{\mathbf{x}}_{#1}}}
\newcommand{\vlam}[1]{\fmtval{\lambda{\color{Black} .}\ \fmtclo{#1}}}
\newcommand{\vapp}[2]{{\color{Black} \fmtne{\mathsf{app}}(\fmtne{#1}, \fmtnf{#2})}}
\newcommand{\vup}[2]{{\color{Black} \uparrow^{\fmtval{#1}} \fmtne{#2}}}
\newcommand{\vnf}[2]{{\color{Black} \downarrow^{\fmtval{#1}} \fmtval{#2}}}
\newcommand{\var}[1]{\fmttm{\mathbf{x}_{\color{Black} #1}}}
\newcommand{\nevar}[1]{\fmtne{\mathbf{x}_{\color{Black} #1}}}
\newcommand{\emp}{()}
\newcommand{\pow}[1]{\mathcal{P}(#1)}
\newcommand{\Term}{\mathbf{Term}}
\newcommand{\Type}{\mathbf{Type}}
\newcommand{\Val}{\mathbf{Val}}
\newcommand{\Cxt}{\mathbf{Cxt}}
\newcommand{\pto}{\rightharpoonup}
\newcommand{\mto}{\xrightarrow{\mathsf{mon}}}
\newcommand{\isctx}[1]{\fmttm{#1}\vdash}
\newcommand{\isterm}[3]{\fmttm{#1}\vdash\fmttm{#2}:\fmttm{#3}}
\newcommand{\isneutral}[3]{\fmttm{#1}\vdash^{\mathsf{neu}}\fmtne{#2}:\fmttm{#3}}
\newcommand{\isnormal}[3]{\fmttm{#1}\vdash^{\mathsf{nf}}\fmtnf{#2}:\fmttm{#3}}
\newcommand{\eqterm}[4]{\fmttm{#1} \vdash \fmttm{#2} \equiv \fmttm{#3} : \fmttm{#4}}
\newcommand{\istype}[2]{\fmttm{#1}\vdash\fmttm{#2}}
\newcommand{\eqtype}[3]{\fmttm{#1}\vdash\fmttm{#2} \equiv \fmttm{#3}}
\newcommand{\subst}[3]{{\color{Black} \fmttm{#1}\{\fmttm{#2}/#3\}}}
\newcommand{\weaken}[2]{{\color{Black} \fmttm{#1}\{\uparrow^{#2}\}}}
\newcommand{\app}[2]{{\color{Black} \fmttm{#1}(\fmttm{#2})}}
\newcommand{\neapp}[2]{{\color{Black} \fmtne{#1}(\fmtnf{#2})}}
\newcommand{\nf}[3]{{\color{Black} \mathsf{nf}(\fmttm{#1}, \fmttm{#2}, \fmttm{#3})}}

\newcommand{\Nf}{\mathcal{N}\!f}
\newcommand{\Ne}{\mathcal{N}e}
\newcommand{\gpheval}[3]{\fmtval{#1}\models\fmttm{#2}\Downarrow\fmtval{#3}}
\newcommand{\gphinst}[3]{\fmtclo{#1}[\fmtval{#2}]\rightsquigarrow\fmtval{#3}}
\newcommand{\gphapp}[3]{\fmtval{#1}\mathbin{@}\fmtval{#2}\rightsquigarrow\fmtval{#3}}
\newcommand{\gphquone}[3]{#1\Vdash\fmtne{#2}\Uparrow\fmttm{#3}}
\newcommand{\gphquotp}[3]{#1\Vdash\fmtval{#2}\Uparrow\fmttm{#3}}
\newcommand{\gphquonf}[3]{#1\Vdash\fmtnf{#2}\Uparrow\fmttm{#3}}
\newcommand{\gphreflectcxt}[2]{{\uparrow}\fmttm{#1}\rightsquigarrow\fmtval{#2}}

\title{Normalization by Evaluation for \\Martin-L{\"of} Type Theory}
\author{Daniel Gratzer}
\date{\today}

\begin{document}
\begin{frame}
  \titlepage
\end{frame}

\begin{frame}
  \frametitle{Goal}

  Produce a function $\nf{\Gamma}{t}{A} : \Cxt \times \Term \times \Type \pto \Term$ so that the following 3 conditions
  hold:
  \begin{enumerate}
  \item $\eqterm{\Gamma}{t_1}{t_2}{A} \iff \nf{\Gamma}{t_1}{A} = \nf{\Gamma}{t_2}{A}$
  \item If $\isterm{\Gamma}{t}{A}$ then $\eqterm{\Gamma}{t}{\nf{\Gamma}{t}{A}}{A}$
  \item If $\isterm{\Gamma}{t}{A}$ then $\nf{\Gamma}{t}{A}$ is a \emph{normal form}\\ -- more on this shortly.
  \end{enumerate}
  \pause
  \centering
  \emph{N.B. is partial as $\Term$ contains ill-typed terms.}
\end{frame}

\begin{frame}
  \frametitle{Why Bother}
  \centering
  Why bother to do this when it's so much easier to not do things?
  \bigskip
  \begin{enumerate}
  \item Lars told me to prove normalization for a type theory
    \pause
  \item Termination, canonicity, consistency are corollaries
  \item Decidability of type-checking\\
    \pause
    This because of the \emph{conversion rule}:
    \[
      \inferrule{
        \eqtype{\Gamma}{A}{B}\\
        \isterm{\Gamma}{t}{A}
      }{
        \isterm{\Gamma}{t}{B}
      }
    \]
    \pause
  \item Adequacy of logical frameworks depends on normalization
  \item Completeness of focused proof strategies is equivalent
  \end{enumerate}
\end{frame}

\begin{frame}
  \frametitle{Why Normalization by Evaluation (NbE)?}
  Techniques for proving normalization abound, why NbE?
  \begin{enumerate}
  \item Scales to support many languages
    \begin{itemize}
    \item full dependent types
    \item proof irrelevant types
    \item impredicative quantification
    \item sized types
    \item (conjectured) guarded dependent type theory
    \item (conjectured) cubical type theory.
    \end{itemize}
  \item Amenable to formalization in a (stronger) type theory
  \item Practical for implementation*
  \item Principled semantic interpretation
  \end{enumerate}
\end{frame}

\begin{frame}
  \frametitle{A Language}
  We need to specify the language that we're going to normalize.

  \pause
  \bigskip

  Sorry.
\end{frame}
\begin{frame}
  \frametitle{A Language}

  \begin{mathpar}
    \inferrule{ }{\isctx{\emp}}\and
    \inferrule{
      \isctx{\Gamma}\\
      \istype{\Gamma}{A}
    }{\isctx{\Gamma.A}}\\
    \inferrule{
      \istype{\Gamma}{A}\\
      \istype{\Gamma.A}{B}
    }{\istype{\Gamma}{A \to B}}\and
    \inferrule{
      \isctx{\Gamma}
    }{\istype{\Gamma}{\Unit} \\ \istype{\Gamma}{\Uni}}\and
    \inferrule{
      \isterm{\Gamma}{A}{\Uni}
    }{\istype{\Gamma}{A}}\\
    \inferrule{
      \isctx{\Gamma}
    }{
      \isterm{\Gamma}{\Unit}{\Uni}\\
      \isterm{\Gamma}{\unit}{\Unit}
    }\and
    \inferrule{
      \istype{\Gamma}{A}\\
      \isterm{\Gamma.A}{t}{B}
    }{\isterm{\Gamma}{\lambda t}{A \to B}}\and
    \inferrule{
      \isterm{\Gamma}{t}{A \to B}\\
      \isterm{\Gamma}{u}{A}
    }{\isterm{\Gamma}{\app{t}{u}}{\subst{B}{u}{0}}}\and
    \inferrule{
      \isctx{\Gamma}
    }{\isterm{\Gamma}{\var{n}}{\weaken{\color{Black} \fmttm{\Gamma}(n)}{n}}}
  \end{mathpar}
\end{frame}

\begin{frame}
  \frametitle{Conventions}
  We use De Bruijn indices ($\var{n}$ points to the $n$th closest binder) for representing
  binding.
  \begin{align*}
    \weaken{t}{i} &\quad \text{``Increment every variable in $t$ by $i$''}\\
    \subst{t}{u}{i} &\quad \text{``Replace $\fmttm{\var{i}}$ with $\fmttm{u}$ in $\fmttm{t}$, adjust variables''}
  \end{align*}
\end{frame}

\begin{frame}
  \frametitle{The Wrinkle}
  The rule which causes so much pain.
  \[
    \inferrule{
      \isterm{\Gamma}{t}{A}\\
      \eqtype{\Gamma}{A}{B}
    }{
      \isterm{\Gamma}{t}{B}
    }
  \]
  Dependence means term equality matters for type equality.
  \[
    \inferrule{
      \eqterm{\Gamma}{A}{B}{\Uni}
    }{\eqtype{\Gamma}{A}{B}}
  \]
\end{frame}
\begin{frame}
  \frametitle{The Wrinkle -- The Main Equality Rules}
  \begin{mathpar}
    \inferrule{
      \isterm{\Gamma}{u}{A}\\
      \isterm{\Gamma.A}{t}{B}
    }{\eqterm{\Gamma}{\app{\color{Black}(\fmttm{\lambda t})}{u}}{\subst{t}{u}{0}}{\subst{B}{u}{0}}}\\
    \inferrule{
      \istype{\Gamma}{A}\\
      \isterm{\Gamma.A}{t}{B}
    }{\eqterm{\Gamma}{\lambda \color{Black}(\app{\weaken{t}{1}}{\var{0}})}{t}{A \to B}}\\
    \inferrule{
      \isterm{\Gamma}{t}{\Unit}
    }{\eqterm{\Gamma}{t}{\unit}{\Unit}}
  \end{mathpar}
\end{frame}

\begin{frame}
  \frametitle{Neutral and Normal Forms}
  Let us isolate special terms which are \emph{canonical} for their equivalence classes. Though this
  can only be observed later.

  \begin{enumerate}
  \item Neutral terms: variables or terms stuck on variables.
  \item Normal forms: terms in $\beta$-normal and $\eta$-long forms.
  \end{enumerate}
  \begin{mathpar}
    \inferrule{
      \isterm{\Gamma}{\var{n}}{A}
    }{\isneutral{\Gamma}{\nevar{n}}{A}}\and
    \inferrule{
      \isneutral{\Gamma}{e}{A \to B}\\
      \isnormal{\Gamma}{v}{A}
    }{\isneutral{\Gamma}{\neapp{e}{v}}{\subst{B}{v}{0}}}\\
    \inferrule{
      \isctx{\Gamma}
    }{
      \isnormal{\Gamma}{\unit}{\Unit}
    }\and
    \inferrule{
      \istype{\Gamma}{A}\\
      \isnormal{\Gamma.A}{t}{B}
    }{\isnormal{\Gamma}{\lambda t}{A \to B}}\and
    \inferrule{
      \isnormal{\Gamma}{A}{\Uni}\\
      \isnormal{\Gamma.A}{B}{\Uni}
    }{\isnormal{\Gamma}{A \to B}{\Uni}}\and
    \inferrule{
      \isneutral{\Gamma}{e}{\Uni}
    }{\isnormal{\Gamma}{e}{\Uni}}\and
  \end{mathpar}
\end{frame}

\begin{frame}
  \frametitle{Normalization by Evaluation}
  \centering
  Now we have a goal, construct $\isnormal{\Gamma}{\nf{\Gamma}{t}{A}}{A}$ given $\isterm{\Gamma}{t}{A}$.
\end{frame}
\begin{frame}
  \frametitle{Normalization by Evaluation -- Historical Context}
  \centering
  Original idea:\\ \textbf{normalize programs using the ambient semantic universe.}

  \bigskip

  Latent in Martin-L\"of's original proofs of the decidability of typing.
\end{frame}

\begin{frame}[fragile]
  \frametitle{Normalization by Evaluation -- Historical Context}
  For instance, code up the following functions:
\begin{lstlisting}
  eval  : Term t -> t
  quote : t -> Term t

  normalize = quote . eval
\end{lstlisting}
  Done in Scheme for the simply-typed lambda calculus at first, adapted to other settings.
\end{frame}

\begin{frame}[fragile]
  \frametitle{Normalization by Evaluation -- Historical Context}
  \begin{centering}
    To adapt to a proof use domains instead of a PL
  \end{centering}
  \[
    D \cong (D \to D) \oplus (\mathbb{N} \cup \mathbb{V})_\bot
  \]
  Then define the following:
  \[
    \mathsf{eval} : \Term \to D \qquad \mathsf{quote} : D \pto \Term
  \]
  \pause
  \bigskip
  N.B. We have shifted from \emph{intrinsic} typing to \emph{extrensic} typing.
\end{frame}

\begin{frame}
  \frametitle{Normalization by Evaluation -- Historical Context}
  These historical approaches are imperfect:
  \begin{itemize}
  \item Intrinsic typing proved intractable for impredicativity or dependent types.
  \item Using domains adds unnecessary complexity and is far removed from implementations.
  \item The direct ``reflect to the metatheory'' approach does not scale to extrensic types.
  \end{itemize}

  \pause
  \bigskip

  Modern NbE uses a different semantic model: syntax.
\end{frame}

\begin{frame}
  \frametitle{A Syntactic Semantic Domain}
  Construct a syntax in which all expressions are canonical.
  \bigskip

  Divided between \fmtne{neutrals}, \fmtnf{normals}, \fmtval{values}, \fmtclo{closures}.
\end{frame}

\begin{frame}
  \frametitle{A Syntactic Semantic Domain -- Neutrals}
  \fmtne{Neutral elements} represent computations which are stuck on some variable.
  \[
    \fmtne{e} ::= \vvar{\ell} \mid \vapp{e}{\vnf{A}{v}}
  \]
  \bigskip

  N.B. The argument to $\vapp{e}{-}$ must be fully evaluated and annotated.
\end{frame}

\begin{frame}
  \frametitle{A Syntactic Semantic Domain -- Closures}
  What happens when we go under a binder?

  \pause
  \bigskip

  We choose to suspend evaluation and record the current state with a \fmtclo{closure}.
  \[
    \fmtclo{f} ::= \mkclo{t}{\rho}
  \]
  $\fmtval{\rho}$ is the \emph{environment} we're interpreting $\fmttm{t}$. This removes the need
  for domains, is called \emph{defunctionalization}.
\end{frame}

\begin{frame}
  \frametitle{A Syntactic Semantic Domain -- Values}
  It's difficult to isolate $\eta$-long forms for dependent type theory.

  We settle for isolating $\beta$-normal forms for now.
  \[
    \fmtval{v}, \fmtval{A} ::=
    \vlam{f} \mid \vunit \mid \vUnit \mid \vuni \mid \vpi{A}{F}
    \only<2>{\mid \vup{A}{e}}
  \]
  \pause
  Need to include \fmtne{neutrals} \emph{with} type information to allow $\eta$-expansions later.

\end{frame}

\begin{frame}
  \frametitle{A Syntactic Semantic Domain}
  \[
    \begin{array}{lcl}
      \fmtval{v}, \fmtval{A} & ::= & \vlam{f} \mid \vunit \mid \vUnit \mid \vuni \mid \vpi{A_1}{F} \mid \vup{A}{e}\\
      \fmtclo{f}, \fmtclo{F} & ::= & \mkclo{t}{\rho}\\
      \fmtne{e} & ::= & \vvar{\ell} \mid \vapp{e}{v}\\
      \fmtnf{n} & ::= & \vnf{A}{v}\\
      \fmtval{\rho} & ::= & \cdot \mid \fmtval{\rho}.\fmtval{v}
    \end{array}
  \]
\end{frame}

\begin{frame}
  \frametitle{Paying the Piper -- Typing Information}
  The usage of $\vnf{A}{v}$ and $\vup{A}{e}$ seems very arbitrary. Why do we need typing
  information?

  \bigskip

  \begin{itemize}
  \item We need type information to know whether $\eta$-expansion is necessary now that we have
    neutrals of all types.\\
    \emph{In the domain-theoretic or intrinsic formulation this was baked in as we disallowed such neutrals.}
    \pause
  \item Coquand proposed adding $\vnf{A}{v}$ to mark a \fmtval{value} that should be $\eta$-expanded
    at type $\fmtval{A}$ during quotation.
  \item Quotation proceeds by casing on this type.
  \end{itemize}
\end{frame}

\begin{frame}
  \frametitle{The Algorithm}
  Now that we have defined our sorts of terms, we can define the algorithm.
  \begin{enumerate}
  \item Evaluate a \fmttm{term} to a \fmtval{value} in some \fmtval{environment}
    \[
      \gpheval{\rho}{t}{v}
    \]
  \item Quote a \fmtnf{normal form} back to a \fmttm{term} in a context of length $c$.
    \[
      \gphquonf{c}{n}{t}
    \]
  \item Inject/reflect a \fmttm{term context} into an \fmtval{environment}.
    \[
      \gphreflectcxt{\Gamma}{\rho}
    \]
  \end{enumerate}
\end{frame}

\begin{frame}
  \frametitle{The Algorithm}

  \begin{align*}
    \nf{\Gamma}{t}{T} = \fmttm{t'} &\iff\\
    &\gphreflectcxt{\Gamma}{\rho} \land{}\\
    &\gpheval{\rho}{t}{v} \land \gpheval{\rho}{T}{A} \land{}\\
    &\gphquonf{\cxtlen{\Gamma}}{\vnf{A}{v}}{t'}
  \end{align*}

  The relational presentation is ideal for a constructive setting.
\end{frame}

\begin{frame}
  \frametitle{The Algorithm -- Defining Evaluation}
  The evaluation judgment is defined by inspection on $\fmttm{t}$.

  \begin{mathpar}
    \inferrule{ }{\gpheval{\rho}{\var{n}}{\rho(n)}}\and
    \inferrule{ }{\gpheval{\rho}{\unit}{\vunit}}\and
    \inferrule{ }{\gpheval{\rho}{\Unit}{\vUnit}}\and
    \inferrule{ }{\gpheval{\rho}{\Uni}{\vuni}}\and
    \inferrule{ }{\gpheval{\rho}{\lambda t}{\vlam{\mkclo{t}{\rho}}}}\and
    \inferrule{
      \gpheval{\rho}{T_1}{A}
    }{\gpheval{\rho}{T_1 \to T_2}{\vpi{A}{\mkclo{T_2}{\rho}}}}\and
  \end{mathpar}
  \pause
  What about the only construct in our language that computes?
\end{frame}

\begin{frame}
  \frametitle{The Algorithm -- Defining Evaluation}
  Application uses an auxiliary relation: $\gphapp{v_1}{v_2}{v}$.
  \begin{mathpar}
    \inferrule{
      \gpheval{\rho.a}{t}{v}
    }{\gphapp{\vlam{\mkclo{t}{\rho}}}{a}{v}}\and
    \inferrule{
      \gpheval{\rho.a}{T}{B}
    }{\gphapp{\vup{\vpi{A}{\mkclo{T}{\rho}}}{e}}{a}{\vup{B}{\vapp{e}{\vnf{A}{a}}}}}\\
    \inferrule{
      \gpheval{\rho}{t}{v_1}\\
      \gpheval{\rho}{u}{v_2}\\
      \gphapp{v_1}{v_2}{v}
    }{\gpheval{\rho}{t(u)}{v}}
  \end{mathpar}

  Rule of thumb:\\
  \centering
  each eliminator gets an auxiliary judgment to either perform $\beta$-reduction or
  construct a new neutral.
\end{frame}

\begin{frame}
  \frametitle{The Algorithm -- Defining Quotation}
  In order to define $\gphquonf{c}{n}{t}$ we need to define two other forms of quotation:
  \begin{itemize}
  \item $\gphquotp{c}{v}{T}$ -- quotation of \fmtval{semantic types}.\\
  \item $\gphquone{c}{e}{t}$ -- quotation of \fmtne{neutrals}.
  \end{itemize}
\end{frame}

\begin{frame}
  \frametitle{The Algorithm -- Defining Quotation}
  Quotation for \fmtnf{normals} proceeds by casing on the \fmtval{type}.
  \begin{mathpar}
    \inferrule{
      \gphapp{v}{\vup{A}{\vvar{c}}}{b}\\
      \gpheval{\rho.\vvar{c}}{T}{B}\\
      \gphquonf{c + 1}{\vnf{B}{b}}{t}
    }{
      \gphquonf{c}{\vnf{\vpi{A}{\mkclo{T}{\rho}}}{v}}{\lambda t}
    }
    \and
    \inferrule{ }{
      \gphquonf{c}{\vnf{\vUnit}{v}}{\unit}
    }
    \\
    \inferrule{ }{\gphquonf{c}{\vnf{\vuni}{\vUnit}}{\Unit}}
    \and
    \inferrule{
      \gphquonf{c}{\vnf{\vuni}{A}}{T_1}\\
      \gpheval{\rho.\vvar{c}}{T}{B}\\
      \gphquonf{c + 1}{\vnf{\vuni}{B}}{T_2}
    }{\gphquonf{c}{\vnf{\vuni}{\vpi{A}{\mkclo{T}{\rho}}}}{T_1 \to T_2}}\and
    \inferrule{
      \gphquone{c}{e}{t}
    }{\gphquonf{c}{\vnf{-}{\vup{-}{e}}}{t}}
  \end{mathpar}
\end{frame}

\begin{frame}
  \frametitle{The Algorithm -- Defining Quotation}
  Quotation for \fmtne{neutrals} proceeds by casing on the \fmtne{neutral} itself.
  \begin{mathpar}
    \inferrule{ }{
      \gphquonf{c}{\vvar{\ell}}{\var{c - (\ell + 1)}}
    }
    \and
    \inferrule{
      \gphquone{c}{e}{t_1}\\
      \gphquonf{c}{n}{t_2}
    }{
      \gphquone{c}{\vapp{e}{n}}{t_1(t_2)}
    }
  \end{mathpar}
  \pause
  Quotation for \fmtval{types} likewise proceed by casing on the \fmtval{type}.
  \begin{mathpar}
    \inferrule{ }{\gphquotp{c}{\vUnit}{\Unit}}
    \and
    \inferrule{ }{\gphquotp{c}{\vuni}{\Uni}}
    \and
    \inferrule{
      \gphquotp{c}{A}{T_1}\\
      \gpheval{\rho.\vvar{c}}{T}{B}\\
      \gphquotp{c + 1}{B}{T_2}
    }{\gphquotp{c}{\vpi{A}{\mkclo{T}{\rho}}}{T_1 \to T_2}}\and
    \inferrule{
      \gphquone{c}{e}{t}
    }{\gphquotp{c}{\vup{-}{e}}{t}}
  \end{mathpar}
\end{frame}

\begin{frame}
  \frametitle{Final Step}
  \begin{enumerate}
  \item \sout{Evaluate a \fmttm{term} to a \fmtval{value} in some \fmttm{environment}}
  \item \sout{Quote a \fmtnf{normal form} back to a \fmttm{term} in a context of length $c$.}
  \item Inject/reflect a \fmttm{term context} into an \fmtval{environment}.
  \end{enumerate}
  \pause
  \begin{mathpar}
    \inferrule{ }{\gphreflectcxt{\emp}{\cdot}}\and
    \inferrule{
      \gphreflectcxt{\Gamma}{\rho}\\
      \gpheval{\rho}{T}{A}\\
    }{\gphreflectcxt{\Gamma.T}{\rho.\vup{A}{\vvar{\cxtlen{\Gamma}}}}}\and
  \end{mathpar}
\end{frame}

\begin{frame}
  \frametitle{Why is This Correct?}

  Now we have to prove some stuff.
  \begin{enumerate}
  \item $\eqterm{\Gamma}{t_1}{t_2}{A} \iff \nf{\Gamma}{t_1}{A} = \nf{\Gamma}{t_2}{A}$
  \item If $\isterm{\Gamma}{t}{A}$ then $\eqterm{\Gamma}{t}{\nf{\Gamma}{t}{A}}{A}$
  \item \only<1>{If $\isterm{\Gamma}{t}{A}$ then $\nf{\Gamma}{t}{A}$ is a normal form}
    \only<2>{\sout{If $\isterm{\Gamma}{t}{A}$ then $\nf{\Gamma}{t}{A}$ is a normal form}}\\
    \only<2>{Can now prove this by induction!}
  \end{enumerate}
\end{frame}


\begin{frame}
  \frametitle{Completeness}
  \[
    \eqterm{\Gamma}{t_1}{t_2}{A} \iff \nf{\Gamma}{t_1}{A} = \nf{\Gamma}{t_2}{A}
  \]
  Proof intuition: build a PER model!
  \begin{itemize}
  \item Each type $\fmtval{A}$ is associated with a PER of \fmtval{values}: $\sem{\fmtval{A}} = R$.
  \item Each PER satisfies the \emph{neutral-normal yoga}
  \end{itemize}
\end{frame}

\begin{frame}
  \frametitle{Completeness -- Neutral-normal yoga}
  Fix two distinguished PERs:
  \begin{align*}
    \Nf &= \{(\fmtnf{n_1}, \fmtnf{n_2}) \mid \forall m.\ \exists \fmttm{t}.\ \gphquonf{m}{n_1}{t} \land \gphquonf{m}{n_2}{t}\}\\
    \Ne &= \{(\fmtne{e_1}, \fmtne{e_2}) \mid \forall m.\ \exists \fmttm{t}.\ \gphquone{m}{e_1}{t} \land \gphquone{m}{e_2}{t}\}\\
  \end{align*}
  For each $R = \sem{\fmtval{A}}$ we require that $R$ is sandwiched between these two PERs.
  \begin{gather*}
    \{(\vup{A}{e_1}, \vup{A}{e_2}) \mid (\fmtne{e_1}, \fmtne{e_2}) \in \Ne\}\\
    {}\subseteq R \subseteq\\
    \{(\fmtval{v_1}, \fmtval{v_2}) \mid (\vnf{A}{v_1}, \vnf{A}{v_2}) \in \Nf\}
  \end{gather*}
\end{frame}


\begin{frame}
  \frametitle{Completeness -- The fundamental lemma}
  We can define a notion of related environments
  $\fmtval{\rho_1} = \fmtval{\rho_2} \in \sem{\fmttm{\Gamma}}$.

  \begin{enumerate}
  \item If $\eqterm{\Gamma}{t_1}{t_2}{T}$ then for all
    $\fmtval{\rho_1} = \fmtval{\rho_2} \in \sem{\fmttm{\Gamma}}$ the following holds.
    \begin{itemize}
    \item $\gpheval{\rho_1}{t_1}{v_1}$
    \item $\gpheval{\rho_2}{t_2}{v_2}$
    \item $\gpheval{\rho_1}{T}{A}$
    \item $\sem{\fmtval{A}} = R$
    \item $(\fmtval{v_1}, \fmtval{v_2}) \in R$
    \end{itemize}
  \item If $\eqtype{\Gamma}{T_1}{T_2}$ then for all
    $\fmtval{\rho_1} = \fmtval{\rho_2} \in \sem{\fmttm{\Gamma}}$ the following holds.
    \begin{itemize}
    \item $\gpheval{\rho_1}{T_1}{A_1}$
    \item $\gpheval{\rho_2}{T_2}{A_2}$
    \item $\sem{\fmtval{A_1}} = \sem{\fmtval{A_2}} = R$
    \item $\forall m.\ \exists \fmttm{T}.\ \gphquotp{m}{A_1}{T} \land \gphquotp{m}{A_2}{T}$
    \end{itemize}
  \end{enumerate}
\end{frame}

\begin{frame}
  \frametitle{Completeness}
  \centering
  the fundamental lemma + neutral-normal yoga = completeness
\end{frame}

\end{document}
