\documentclass[usenames,dvipsnames,aspectratio=169,12pt]{beamer}
\usepackage{tikz}
\usetikzlibrary{tikzmark,positioning}
\usepackage[customcolors]{hf-tikz}
\usepackage{natbib}
\usepackage{bibentry}
\usepackage{microtype, times}
\usepackage{amsmath, amssymb, mathpartir, stmaryrd}
\usepackage{wasysym}
\usepackage{textcomp}
\usepackage[normalem]{ulem}
\usepackage{listings}
\usepackage{color}
\usepackage{xparse}

\NewDocumentCommand{\Arr}{m m}{#1 \to #2}
\NewDocumentCommand{\Fun}{m m}{\mathsf{fun}\, #1 \Rightarrow #2}
\NewDocumentCommand{\App}{m m}{#1(#2)}
\NewDocumentCommand{\Nat}{}{\mathsf{Nat}}
\NewDocumentCommand{\Zero}{}{\mathsf{zero}}
\NewDocumentCommand{\Succ}{m}{\mathsf{succ}(#1)}
\NewDocumentCommand{\Vect}{m m}{\mathsf{Vec}(#1, #2)}
\NewDocumentCommand{\Nil}{}{\mathsf{nil}}
\NewDocumentCommand{\Cons}{m m}{#1 {:} #2}

\usefonttheme[onlymath]{serif}
\definecolor{JadeGreen}{RGB}{0,168,107}
\definecolor{MunsellPurple}{RGB}{159,0,197}
\definecolor{LightGrey}{gray}{0.9}
\definecolor{CobaltBlue}{RGB}{0,71,171}
\definecolor{FireBrick}{RGB}{228,34,23}
\definecolor{Alabaster}{RGB}{250,250,250}

\hfsetfillcolor{blue!15}
\hfsetbordercolor{blue!15}
\renewcommand<>{\hl}[2]{\only#3{\tikzmarkin{#2}}{#1}\only#3{\tikzmarkend{#2}}}

\setbeamertemplate{navigation symbols}{}
\setbeamertemplate{footline}{%
  \raisebox{7pt}{\makebox[\paperwidth]{\hfill\makebox[10pt]{\scriptsize\insertframenumber}\quad}}}
\setbeamertemplate{headline}{}
\setbeamercolor{example text}{fg=CobaltBlue}
\setbeamertemplate{itemize items}{$\bullet$}

\title{Implementing a (Modal) Dependent Type Theory}
\author{\textbf{Daniel Gratzer}\inst{1}\\%
  Jonathan Sterling\inst{2} \and Lars Birkedal\inst{1}}
\institute{\inst{1} This University \smiley{} \\ \inst{2} Not This University \frownie{}}
\date{May 24 2019}

\begin{document}
\begin{frame}[noframenumbering]
  \titlepage
\end{frame}

\begin{frame}
  \frametitle{Hello!}
  First things first, the title of this talk is a lie.
  \pause

  \begin{itemize}
  \item I don't want to talk about \emph{modal} dependent type theory.
  \item I'd like to talk about the core idea behind this work.
  \end{itemize}
  \pause
  \bigskip

  An alternative title:
  \begin{center}
    \color{Blue}
    \Large
    Implementing type theory
  \end{center}

\end{frame}

\begin{frame}
  \frametitle{Setting the Scene}

  What is type theory? Type theory is a....
  \only<1>{\setbeamercolor{itemize item}{fg=red}}
  \begin{itemize}
  \item well-behaved programming language with a rich type system.
    \only<1>{\setbeamercolor{itemize item}{fg=CobaltBlue}}
    \only<2>{\setbeamercolor{itemize item}{fg=red}}

  \item framework for reasoning about mathematical objects.
    \only<2>{\setbeamercolor{itemize item}{fg=CobaltBlue}}
    \only<3>{\setbeamercolor{itemize item}{fg=red}}

  \item foundation for mathematics.
  \end{itemize}
  \pause\pause\pause
  \bigskip


  Each viewpoint induces requirements on a good type theory is.
\end{frame}

\begin{frame}
  \frametitle{The Fundamental Tension}

  We want a type theory which is suitable for

  \begin{center}
    \textcolor{Red}{studying mathematics}
  \end{center}

  but at the same time still serves as setting for

  \begin{center}
    \textcolor{Blue}{programming}
  \end{center}

\end{frame}

\begin{frame}
  \frametitle{A Concrete Issue}

  Where does this tension manifest itself?

  \begin{enumerate}
  \item Types represent \textcolor{Red}{mathematical objects}
  \item \textcolor{Blue}{Type-checking} must be implementable by a machine
  \end{enumerate}

  When should two terms or types be considered equal by the system?
\end{frame}

\begin{frame}
  \frametitle{Some Background}
  I'd like to explain this problem more precisely. First we need some types.
\end{frame}

\begin{frame}
  \frametitle{Two Standard Types}
  Type theory has functions:
  \[
    \begin{array}{ccc}
      \Arr{A}{B} & \Fun{x}{M} & \App{f}{a} \\
      & \App{(\Fun{x}{x})}{M} = M &
    \end{array}
  \]

  \bigskip

  Type theory has natural numbers
  \[
    \Nat \quad \Zero \quad \Succ{N} \quad N * M \quad  N + M
  \]
\end{frame}

\begin{frame}
  \frametitle{A More Exotic Type}

  Type theory also has \emph{length-indexed} lists: vectors.
  \[
    \Vect{A}{N} \quad \Nil \quad \Cons{M}{L}
  \]

\end{frame}

\end{document}
